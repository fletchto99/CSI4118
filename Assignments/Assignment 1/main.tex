\documentclass[fleqn, 12pt]{article}

% packages %
\usepackage[includeheadfoot,headheight=15pt,margin=0.5in,left=1in,right=1in,headsep=10pt]{geometry} % page margins %
\usepackage{mathtools, amssymb} % math %
\usepackage{tabularx, multirow} % tables %
\usepackage{minted} % code %
\usepackage{graphicx} % graphics %
\usepackage{enumerate} % lists %
\usepackage{adjustbox} % images %
\usepackage[T1]{fontenc} % fonts %
\usepackage[protrusion=true,expansion=true]{microtype} % font clarity %
\usepackage{fancyhdr} % headers and footers %
\usepackage{lastpage} % reference page numbers %
\usepackage{color} % colors for code %
\usepackage{tikz} % for graphs %
\usepackage{soul} % for strikethroughout %
\usepackage{upquote} % Fix single quotes %
\usepackage{etoolbox} % Conditional checks %
\usepackage{hyperref} % Hyperlinks %
\usepackage{indentfirst} % fix indentation - only for essays %
\usepackage[figure,table]{totalcount} % For counting tables and figures %
\usepackage[utf8]{inputenc} % Encode as UTF-8 %
\usepackage{biblatex} % References %
\addbibresource{references.bib} % bib source %

% Document details %
\newcommand{\university}{University of Ottawa}
\newcommand{\name}{Matthew Langlois}
\newcommand{\studentNumber}{7731813}
\newcommand{\semester}{Fall 2017}
\newcommand{\assignmentType}{Assignment}
\newcommand{\assignmentNumber}{1}
\newcommand{\dueDate}{Oct 12}
\newcommand{\courseCode}{CSI4118}
\newcommand{\courseTitle}{Networking Protocols}
\newcommand{\essayTitle}{<Title>} % only for essays %
\newcommand{\essaySubtitle}{<subtitle>} % only for essays %
\newcommand{\essayAbstract}{} % Only for essays -- leave empty for no abstract %

% Center image and diagrams %
\adjustboxset*{center}

% Code settings %
\setminted{
    fontfamily=tt,
    gobble=0,
    frame=single,
    funcnamehighlighting=true,
    tabsize=4,
    obeytabs=false,
    mathescape=false
    samepage=false,
    showspaces=false,
    showtabs =false,
    texcl=false,
    breaklines=true,
}

% inline code %
\definecolor{codegray}{gray}{0.9}
\newcommand{\code}[1]{\colorbox{codegray}{\texttt{#1}}}

% Code from tile %
\newcommand{\codefile}{\inputminted}

% Graphing stuff %
\usetikzlibrary{arrows.meta}
\usetikzlibrary{positioning}
\usetikzlibrary{matrix}
\usetikzlibrary{automata}

% Define ceiling and floor functions %
\DeclarePairedDelimiter\ceil{\lceil}{\rceil}
\DeclarePairedDelimiter\floor{\lfloor}{\rfloor}

% Create set compliment command %
\newcommand{\setcomp}[1]{{#1}^{\mathsf{c}}}

% Create logic command aliases %
\newcommand{\limplies}{\rightarrow}
\newcommand{\nequiv}{\not\equiv}
\newcommand{\liff}{\leftrightarrow}

% first page header & footer %
\fancypagestyle{assignment}{
    \fancyhf{}
    \renewcommand{\footrulewidth}{0.1mm}
    \fancyfoot[R]{\assignmentType\text{ }\assignmentNumber}
    \fancyfoot[C]{\thepage \hspace{1pt} of \pageref{LastPage}}
    \fancyfoot[L]{\courseCode\text{ }\semester}
    \renewcommand{\headrulewidth}{0mm}
}

% Frontmatter for essay page numbering%
\fancypagestyle{frontmatter}{
    \fancyhf{}
    \renewcommand{\footrulewidth}{0.1mm}
    \fancyfoot[R]{\assignmentType\text{ }\assignmentNumber}
    \fancyfoot[C]{\thepage \hspace{1pt} of \pageref{EndFrontMatter}}
    \fancyfoot[L]{\courseCode\text{ }\semester}
    \fancyhead[L]{\name}
    \fancyhead[R]{\studentNumber}
}

% Essay body page numbering %
\fancypagestyle{body}{
    \fancyhf{}
    \renewcommand{\footrulewidth}{0.1mm}
    \fancyfoot[R]{\assignmentType\text{ }\assignmentNumber}
    \fancyfoot[C]{\thepage \hspace{1pt} of \pageref{LastPage}}
    \fancyfoot[L]{\courseCode\text{ }\semester}
    \fancyhead[L]{\name}
    \fancyhead[R]{\studentNumber}
}

% Page header and footers %
\fancyhf{}
\renewcommand{\footrulewidth}{0.1mm}
\fancyfoot[R]{\assignmentType\text{ }\assignmentNumber}
\fancyfoot[C]{\thepage \hspace{1pt} of \pageref{LastPage}}
\fancyfoot[L]{\courseCode\text{ }\semester}
\fancyhead[L]{\name}
\fancyhead[R]{\studentNumber}

% Apply headers & footer page style %
\pagestyle{fancy}

% Assignment first page header %
\newcommand{\beginassignemnt}{
    % Prevent paragraph indents, this isn't an English assignment! %
    \newlength\tindent
    \setlength{\tindent}{\parindent}
    \setlength{\parindent}{0pt}
    
    \thispagestyle{assignment}
    \noindent
    \courseTitle \hfill \university\\
    \courseCode \hfill \semester
    \begin{center}
        \textbf{\assignmentType\text{ }\#\assignmentNumber}\\
        \name \hspace{1pt} - \studentNumber\\
        \dueDate\\
    \end{center}
    \vspace{6pt}
    \hrule
    \vspace{1.5\headsep}
}

% Essay titlepage stuff %
\newcommand{\beginessay}{
    % Load all citations %
    \nocite{*}
    
    % Special numbering for front matter %
    \pagestyle{frontmatter}
    \pagenumbering{roman}
    
    % Titlepage stuff %
    \begin{center}
        \normalsize 
        \textsc{\university}\\[5cm]
        \LARGE \textbf{\MakeUppercase{\essayTitle}}\\[0.5cm]
        \large \text{ }\essaySubtitle\text{ }\\[10cm] % blank \texts are used for empty subtitles %
        \normalsize
        \textsc{\name}\\
        \textsc{\studentNumber}\\ 
        \textsc{\courseCode}\\
        \textsc{\semester}\\
        \textsc{\dueDate}
    \end{center}
    \thispagestyle{empty}
    % End title page stuff %
    
    % Table of Contents %
    \newpage
    \tableofcontents
    \newpage
    
    % If more than 1 table/figure show appropriate lists %
    \iftotalfigures
        \addcontentsline{toc}{section}{\listfigurename}
        \listoffigures
    \fi
    \iftotaltables
        \addcontentsline{toc}{section}{\listtablename}
        \listoftables
    \fi
    
    % Display an abstract if the abstract isn't empty %
    \ifdefempty{\essayAbstract}{}{
        \newpage
        \addcontentsline{toc}{section}{Abstract}
        \begin{abstract}
            \essayAbstract
        \end{abstract}
        
    }
    \label{EndFrontMatter}
    \newpage
    
    % Reset page numbering %
    \pagenumbering{arabic}
    \pagestyle{body}
}

% Update the bibliography command to add its self to the references
\let\oldprintbib\printbibliography
\renewcommand{\printbibliography}{
    \newpage
    \oldprintbib
    \addcontentsline{toc}{section}{References}
    \newpage
}

% Fixes a Pygments bug in ES6 -- Thanks ShareLatex! %
\makeatletter
\expandafter\def\csname PYGdefault@tok@err\endcsname{\def\PYGdefault@bc##1{{\strut ##1}}}
\makeatother

% Begin the document %
\begin{document}

% makes the header for assignment/titlepage for essay %
% \beginessay
\beginassignemnt

\section*{Part One}

\begin{enumerate}
    \item 
        Since we know the address is 193.1.1.0 which is class C then the subnet mask 255.255.255.0 is applied. This means we can modify the last octet in the mask to modify the network which gives us 256 possible addresses. By taking 2 bits of the last octet we can apply a 193.1.1.0/26 mask (255.255.255.192) to get 4 separate networks (.0, .64, .128, .192). 
        
        \begin{tabular}{|c|c|c|c|}
            \hline
             & Router & Device 1 & Device 2 \\\hline
             Network 1 & 193.1.1.0 & 193.1.1.1 & 193.1.1.2\\\hline
             Network 2 & 193.1.1.64 & 193.1.1.65 & 193.1.1.66\\\hline
             Network 3 & 193.1.1.128 & 193.1.1.129 & 193.1.1.130\\\hline
             Network 4 & 193.1.1.192 & 193.1.1.193 & 193.1.1.194\\\hline
        \end{tabular}
    \item 
        In each of these questions the mask is applied to the first n-bits of the ip address to determine the subnet. Once the subnet is determined we can look up the address in the routing table to determine which device to route the packet to next. 
        \begin{enumerate}
            \item 
                Taking the first 22 bits of 135.46.63.10 you get 135.46.60.0. Thus this matches Interface 1 which is where the packet will be forwarded to.
            \item
                Taking the first 22 bits of 135.46.57.14 you get 135.45.56.0. Thus this matches Interface 0 which is where the packet will be forwarded to.
            \item
                Taking the first 23 bits of 135.46.52.2 you get 135.45.52.0. This does not match any of the subnets in the routing table thus the packet will be forwarded to the default gateway which in this case is Router 2.
            \item
                Taking the first 23 bits of 192.53.40.7 you get 192.53.40.0. Thus this matches Router 1 which is where the packet will be forwarded to.
            \item
                Taking the first 23 bits of 192.53.56.7 you get 192.53.56.0. This does not match any of the subnets in the routing table thus the packet will be forwarded to the default gateway which in this case is Router 2
        \end{enumerate}
        
        \newpage
    \item
        Dijkstra's Algorithm - Each iteration the values in the table are updated if there is a shorter path found to the node which already exists in the cloud.
    
        \begin{tabular}{|c|c|c|c|c|c|c|c|}
        \hline
             Iter & B & C        & D        & E        & F        & G & H  \\\hline
                1 & 2 & $\infty$ & $\infty$ & $\infty$ & $\infty$ & 6 & $\infty$ \\\hline
                2 & 2 & 9        & $\infty$ & 4        & $\infty$ & 6 & 10       \\\hline
                3 & 2 & 9        & $\infty$ & 4        & 6        & 5 & 10       \\\hline
                4 & 2 & 9        & 10       & 4        & 6        & 5 & 8        \\\hline
        \end{tabular}
        
    \item
        
        \begin{enumerate}[a)]
            \item
                The window size is constantly increased by 1, until the maximum size is hit. In this case we need to transmit 2000 packets of 1KB with the max window size of 500KB. To transmit the packets the window will increase by 1, until all 2000 packets are sent ($1+2+3+...+n=2000$) or $\frac{n(n+1)}{2} = 2000$
                
                Solving for only the positive side of n (since it is a quadratic function) we get 63. Thus the congestion window size of 500KB is never reached so no decrease occurs. Therefore the total time is $63 \cdot 3.3 = 207$ which is 207 ms.
                
            \item
                The window size, starting at 1KB, is doubled, until the maximum size is hit. To fit all 2000 packets we will need to make 11 trips, which is less than the 500KB maximum window size: $1+2+4+8+16+32+64+128+256+512+1024 = 2047$

                Thus the total time is $11 \cdot 3.3 = 36.3$ which is 36.3ms.
        \end{enumerate}
\end{enumerate}

\newpage

\section*{Part Two}

\begin{enumerate}[1)]
    \item
        The IP address was found in the IPV6 frame which come right after the ethernet 2 header and before the UDP datagram. The destination address was \code{fe80::b8b5:c1ff:fe9d:9eb4}. The destination address can be seen in the following image:\\
        
        \adjincludegraphics[width=\textwidth]{DNS.png}
        
    \item
        The two types of messages being exchanged are ICMP request (Echo ping request) and ICMP response (Echo ping response). The ping request is sending a request to the server with a TTL of 64. The server then responds with a TTL of 49. Therefore the server is reachable after 15 hops. The messages being exchanged can be seen in the image below:\\
        
        \adjincludegraphics[width=\textwidth]{messages.png}
        
    \newpage
    \item
        The IP address of bbc.com is \code{212.58.246.79} as seen in the ping request. Refer to the image in Q4 the line "destination" is where the ping request is being routed to.
        
    \item
        The IP address of my personal computer is \code{192.168.1.188} as seen in the ping request. Refer to the image... the line "source" is where the ping request is being sent from (my computer).\\
        
        \adjincludegraphics[width=\textwidth]{ping.png}
\end{enumerate}

\end{document}